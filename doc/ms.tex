%%%%%%%%%%%%%%%%%%%%%%%%%%%%%%%%%%%%%%%%%%%%%%%%%%%%%%%%%%%%%%%%%%%%%%%%%%%%%%
%
% All Ze Lenses
%
%%%%%%%%%%%%%%%%%%%%%%%%%%%%%%%%%%%%%%%%%%%%%%%%%%%%%%%%%%%%%%%%%%%%%%%%%%%%%%

\documentclass[useAMS,usenatbib]{mn2e}

\voffset=-0.8in

% Packages:
\usepackage{graphicx}
\usepackage{amsmath}
\usepackage{xspace}

% Macros:
\input{macros.tex}
\def\ucsb{Department of Physics, University of California, Santa Barbara, CA 93106, USA}
\def\kipac{Kavli Institute for Particle Astrophysics and Cosmology, Stanford University, 452 Lomita Mall, Stanford, CA 94035, USA}
\def\thirdplace{Institute 3, University 3, City 3, Country 3}
\def\fourthplace{Institute 4, University 4, City 4, Country 4}
\def\fifthplace{Institute 5, University 5, City 5, Country 5}
\def\sixthplace{Institute 6, University 6, City 6, Country 6}

\def\email{\tt sonnen@physics.ucsb.edu}


%%%%%%%%%%%%%%%%%%%%%%%%%%%%%%%%%%%%%%%%%%%%%%%%%%%%%%%%%%%%%%%%%%%%%%%%%%%%%%

\title[Cosmology with Large Samples of Lenses]
{Accurate Cosmological Inferences from Large Samples of Time Delay Gravitational Lenses}

\author[All of us]{%
  Alessandro Sonnenfeld,$^{1}$\thanks{\email}
  Phil Marshall,$^{2}$
  Third Author,$^{3}$
\newauthor{%
  Fourth Author,$^{4}$
  Fifth Author,$^{5}$
  Sixth Author.$^{6}$}
  \medskip\\
  $^1$\ucsb\\
  $^2$\kipac\\
  $^3$\thirdplace\\
  $^4$\fourthplace\\
  $^5$\fifthplace\\
  $^6$\sixthplace\\
}

%%%%%%%%%%%%%%%%%%%%%%%%%%%%%%%%%%%%%%%%%%%%%%%%%%%%%%%%%%%%%%%%%%%%%%%%%%%%%%

\begin{document}

\date{To be submitted to MNRAS}

\pagerange{\pageref{firstpage}--\pageref{lastpage}}\pubyear{2015}

\maketitle

\label{firstpage}

%%%%%%%%%%%%%%%%%%%%%%%%%%%%%%%%%%%%%%%%%%%%%%%%%%%%%%%%%%%%%%%%%%%%%%%%%%%%%%

\begin{abstract}

Time delay distance cosmography is a promising cosmographic probe. Stage III and
IV Dark Energy surveys (such as those planned with DES and LSST) will contain
samples of hundreds of gravitationally-lensed quasars and supernovae, which,
when followed up with high resolution imaging, should each yield a measurement
of distance in the universe that is precise to 5-10\%. In this paper we explore
the hierarchical inference of cosmological parameters from toy simulated lens
samples, investigating the impact of various simplifying assumptions on the
accuracy achieved in the joint analysis. We find that to realize the available
sub-percent accuracy on the Hubble constant, the individual lenses' measurements
must be combined quite carefully: first marginalizing over individual system's
model parameters and then multiplying the marginal likelihoods can introduce
biases of up to [X\%]. In a full hierarchical inference (carried out by
importance sampling of interim MCMC chains), we find that the cosmology accuracy
is [more/less] sensitive to simple assumptions about the lens model than it is
the structure of the conditional PDFs for those lens models' parameters, and
that working with lens models that are {\it more}  flexible than the true model
incurs [only small/quite large] losses in precision. Finally, we find that a
parent ensemble that is three times larger than the anticpated LSST ``Gold''
sample of lenses contains enough information to increase the precision by [a
factor of X], even when no high resolution follow-up data is available.

\end{abstract}

\begin{keywords}
  Need keywords.
\end{keywords}

%%%%%%%%%%%%%%%%%%%%%%%%%%%%%%%%%%%%%%%%%%%%%%%%%%%%%%%%%%%%%%%%%%%%%%%%%%%%%%

\section{Introduction}

Goes here.

We are interested in the following questions:

\begin{itemize}

\item First question?

\item Second question?

\item Third question?

\end{itemize}

This paper is organized as follows. In \Sref{sec:first} we \ldots
In \Sref{sec:discuss} we present our conclusions.


%%%%%%%%%%%%%%%%%%%%%%%%%%%%%%%%%%%%%%%%%%%%%%%%%%%%%%%%%%%%%%%%%%%%%%%%%%%%%%

\section{Next section}
\label{sec:next}


%%%%%%%%%%%%%%%%%%%%%%%%%%%%%%%%%%%%%%%%%%%%%%%%%%%%%%%%%%%%%%%%%%%%%%%%%%%%%%

\section{Conclusions}
\label{sec:conclude}

Summarize briefly.

Our conclusions regarding \ldots can be stated as follows:

\begin{itemize}

\item First answer.

\item Second answer.

\item Third answer.

\end{itemize}

Wrap up.


%%%%%%%%%%%%%%%%%%%%%%%%%%%%%%%%%%%%%%%%%%%%%%%%%%%%%%%%%%%%%%%%%%%%%%%%%%%%%%

\section*{Acknowledgements}

\input{acknowledgments.tex}

%%%%%%%%%%%%%%%%%%%%%%%%%%%%%%%%%%%%%%%%%%%%%%%%%%%%%%%%%%%%%%%%%%%%%%%%%%%%%%
% MNRAS does not use bibtex, input .bbl file instead. Generate this in the
% makefile using bubble script in scriptutils:


%%%%%%%%%%%%%%%%%%%%%%%%%%%%%%%%%%%%%%%%%%%%%%%%%%%%%%%%%%%%%%%%%%%%%%%%%%%%%%

\label{lastpage}
\bsp

\end{document}

%%%%%%%%%%%%%%%%%%%%%%%%%%%%%%%%%%%%%%%%%%%%%%%%%%%%%%%%%%%%%%%%%%%%%%%%%%%%%%
